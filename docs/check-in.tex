\documentclass[10pt]{article}
\renewcommand{\rmdefault}{phv} % Arial
\renewcommand{\sfdefault}{phv} % Arial
\renewcommand{\baselinestretch}{1.5}
\usepackage{fullpage}
\begin{document}
\title{BIO 321G Project Check-in: Invasive Species in Avida}
\author{Sarah Dean, Chinedum Njoku, Lane Smith}
\maketitle
Outline: Each group member will run three instances of the Avida program: a)
growing the 'native' species; b) growing the 'invasive' species; c) staging the
invasion of the the invasive species into the environment of the native
species. All group members will use the same configuration files appropriate
for each instance, with each instance seeded with the digits of the member's UT
EID. This will provide for a 3 independent trials of the experiment and an
adequate amount of data. The data will then be analyzed as to determine whether
the invasive species did have a significant advantage that allowed them to take
over the environment of the native species.

Current Status: The documentation has been read and the appropriate
configuration files created for all but the 'invasion'. Each member should have
run their instances for 'growing' the two species by Tuesday night. 

Our group would like suggestions on which metrics would be most appropriate for
measuring whether the invasive species was successful in taking over a
significant portion of the environment. More specifically, which triggers
should be written into the events.cfg file that will provide us with useful
data upon completing our analysis? Furthermore, we would like to discuss the
statistical tests and the feasible statistical power that we can hope to
achieve. Also to be discussed are possible design flaws and the chance that the
'invasion' does not produce results. 


\end{document}
